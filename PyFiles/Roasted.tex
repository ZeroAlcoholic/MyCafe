\documentclass[10pt,a4paper]{article}
\usepackage[usenames,dvipsnames]{color}
\usepackage{pdfpages}
\usepackage{array,hyperref , pifont, amsmath, amsthm, amssymb, tikz, enumerate,natbib}
\usepackage[CJKspace = true]{xeCJK}
\usepackage{tabularx}
\usepackage{bibentry} %try to show the full cieation
%\usepackage{datetime}
%\usdate
%\bibliographystyle{abbrv}
%\nobibliography*
%----------------orig79---77---------------0.27cm---orig0.33-----------
\usepackage[left=-0.645cm, right=1.07cm, top=3.43cm, bottom=0.45cm]{geometry}
%\usepackage[left=-0.78cm, right=0.04cm, top=1.38cm, bottom=1.3cm]{geometry}
\renewcommand{\baselinestretch}{1.1} 
\newcommand{\tabincell}[2]{\begin{tabular}{@{}#1@{}}#2\end{tabular}}
%\newcolumntype{C}[1]{>{\centering\arraybackslash}p{#1}}
\renewcommand{\baselinestretch}{1.22}
\usepackage{datenumber}
\setdatetoday
\usepackage{frcursive}
\usepackage[T1]{fontenc}
\usepackage{nopageno}%\usepackage[orig, british, cleanlook]{isodate}
\setCJKmainfont{標楷體} %\XeTeXlinebreaklocale "zh" 
\renewcommand{\today}{\ifnum\number\day<10 0\fi \number\day \space%
\ifcase \month \or  Jan\or Feb\or Mar\or Apr\or May\or Jun\or
   Jul\or Aug\or Sep\or Oct\or Nov\or Dec\fi,%
\number \year} 
\usepackage{advdate}
\begin{document} 
\setdatetoday %

\pagestyle{empty}
\newcolumntype{C}[1]{%
%-----------2.98cm---3cm-for-big-------原始2.63cm-----------
 >{\vbox to 2.88cm\bgroup\vfill\centering}%
 p{#1}%
 <{\egroup}}  
%                    0.345em
%--------------------0.93em-----------

\begin{tabular}{C{0.31\textwidth} C{0.36\textwidth} C{0.28\textwidth}}
\shortstack[c]{
%%11%%
}& \shortstack[c]{
%%12%%
}& \shortstack[c]{
%%13%%
}\vspace*{-0.1em}\tabularnewline
\shortstack[c]{
%%21%%
}& \shortstack[c]{
%%22%%
}& \shortstack[c]{
%%23%%
}\vspace*{-0.1em}\tabularnewline
\shortstack[c]{
%%31%%
}& \shortstack[c]{
%%32%%
}& \shortstack[c]{
%%33%%
}\vspace*{-0.1em}\tabularnewline
\shortstack[c]{
%%41%%
}& \shortstack[c]{
%%42%%
}& \shortstack[c]{
%%43%%
}\vspace*{-0.1em}\tabularnewline
\shortstack[c]{
%%51%%
}& \shortstack[c]{
%%52%%
}& \shortstack[c]{
%%53%%
}\vspace*{-0.1em}\tabularnewline
\shortstack[c]{
%%61%%
}& \shortstack[c]{
%%62%%
}& \shortstack[c]{
%%63%%
}\vspace*{-0.1em}\tabularnewline
\shortstack[c]{
%%71%%
}& \shortstack[c]{
%%72%%
}& \shortstack[c]{
%%73%%
}\vspace*{-0.1em}\tabularnewline
\end{tabular}
\end{document}

%\hspace*{-0.25em} { \vspace*{0.25cm} 
%當季特選  \small 低溫烘焙 %3%
% %4%
%}\\
%{ 
%\LARGE \hspace*{0.50em} \vspace*{-1.4cm}  
%\hspace*{0.3em}\color{Sepia} 蘋果果乾%5%
%}\vspace*{-0.05em}\\
%{\hspace*{-12em}\includegraphics[trim=0 6em 0 -6em,width=0.068\textwidth]{NewC.png}}\vspace*{-4.0em}\\
%\hspace*{13em}
%\includegraphics[trim=0 -3.3em 0 0,width=0.065\textwidth]{NewQR.png}  
%\vspace*{-0.55em}\\
%\setdatetoday \addtocounter{datenumber}{
%0%date%
%}\setdatebynumber{\thedatenumber}
%{\scriptsize 烘焙日:{\scriptsize\datedate}\hspace{0.3em}% 
%烘焙者: \textbf{\textcursive{Chiao-Ching}}}\vspace*{-0.05em}\\
% \addtocounter{datenumber}{
% %Feed
%0 }%
%\setdatebynumber{\thedatenumber} 
%{\small
% 無添加%Raost%
%\enskip \scriptsize 建議賞味:} \scriptsize{\datemonthname ~\thedateday}$\thicksim$\addtocounter{datenumber}{45}\setdatebynumber{\thedatenumber} \datedate
%
%
%
%